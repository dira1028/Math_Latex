% !TeX root = main.tex
\section{562 to 567}

\noindent \textbf{[Stirling's approximation]}
\begin{align*}
    n!               = \sqrt{2\pi n} \left( \frac{n}{e} \right)^n e^{r_n} \\
    \intertext{여기서 $r_n$은 다음 부등식을 만족한다.}
    \frac{1}{12n+1}  < r_n < \frac{1}{12n}
\end{align*}
\begin{proof}
    Let\\
    \[S_n=\log(n!)=\sum_{p=1}^{n-1}\log(p+1)\]
    and write
    \[log(p+1)=A_p+b_p-\epsilon_p\]
    where
    \begin{align*}
        A_p        & = \int_{p}^{p+1} \log x \, dx,                                              \\
        b_p        & = \frac{1}{2} \big( \log(p+1) - \log p \big),                               \\
        \epsilon_p & = \int_{p}^{p+1} \log x \, dx - \frac{1}{2} \big[ \log(p+1) + \log p \big].
    \end{align*}

    then
    \begin{align*}
        S_n        & =\sum_{p=1}^{n-1}(A_p+b_p-\epsilon_p)=\int_{1}^{n}\log(x)dx+\frac{1}{2}\log(n)-\sum_{p=1}^{n-1}\epsilon_p \\
                   & =(n+\frac{1}{2}\log(n)-n+1)-\sum_{p=1}^{n-1}\epsilon_p,                                                   \\
        \epsilon_p & =\frac{2p+1}{2}\log(\frac{p+1}{p})-1.
    \end{align*}
    using Taylor expansion of $\log(1+x)$, we have
    \begin{align*}
        \log(\frac{1+x}{1-x})=2(x+\frac{x^3}{3}+\frac{x^5}{5}+\cdots)
    \end{align*}
    valid for $|x|<1$, let $x=(2p+1)^{-1}$, so that $\frac{1+x}{1-x}=\frac{p+1}{p}$.
    \begin{align*}
        \epsilon_p=\frac{1}{3(2p+1)^2}+\frac{1}{5(2p+1)^4}+\cdots
    \end{align*}


    \noindent \textbf{Upper bound of $\epsilon_p$:} \\
    Since the coefficients decrease ($\frac{1}{3} > \frac{1}{5} > \frac{1}{7} > \cdots$),
    \begin{align*}
        \epsilon_p < & \frac{1}{3}x^2 (1 + x^2 + x^4 + \cdots)                                    \\
                     & =\frac{x^2}{3}\cdot\frac{1}{1-x^2}=\frac{1}{12}(\frac{1}{p}-\frac{1}{p+1})
    \end{align*}

    \noindent \textbf{lower bound of $\epsilon_p$:} \\
    \begin{align*}
        \epsilon_p>\frac{x^2}{3}(1+\frac{x^2}{3}+\frac{x^4}{3^2})=\frac{x^2}{3}\cdot\frac{1}{1-\frac{x^2}{3}}=\frac{1}{12p^2+12p+2}>\frac{1}{12p^2+14p+\frac{13}{12}}
    \end{align*}

    rewrite:
    \begin{align*}
        \sum_{p=1}^{n-1}\epsilon_p & = \underbrace{\sum_{p=1}^{\infty}\epsilon_p}_{B} - \underbrace{\sum_{p=n}^{\infty}\epsilon_p}_{r_n} \\
        \frac{1}{13}               & <B<\frac{1}{12}                                                                                     \\
        \frac{1}{12n+1}            & <r_n<\frac{1}{12n}                                                                                  \\
    \end{align*}
    rewrite $S_n$:
    \begin{align*}
        S_n & = \left(n+\frac{1}{2}\right)\log n - n + 1 - B + r_n   \\
        \intertext{so,}
        n!  & = e^{S_n}                                              \\
            & = e^{1-B} \cdot n^{n+\frac{1}{2}} e^{-n} \cdot e^{r_n}
    \end{align*}
    and $e^{1-B}=\sqrt{2\pi}$.

\end{proof}




\section{CLT}

\begin{theorem}[Central Limit Theorem]
    assume that the random variables $X_1, X_2, \ldots$ are i.i.d with expectation $\mu$ and variance $\sigma^2$. Then:
    \[P\left(\frac{X_1+X_2+\cdots+X_n-\mu n}{\sigma \sqrt{n}} \leq x \right) \to P(Z\leq x)\]
\end{theorem}
\begin{proof}
    let $$T_n = \frac{S_n-n\mu}{\sigma \sqrt{n}}$$\\
    we will show $\phi_{T_n}(t) \to \phi_Z(t).$\\
    let $Y_i=\frac{X_i-\mu}{\sigma}$, then $E[Y_i]=0$, $Var(Y_i)=1$\\
    \begin{align*}
        \phi_{T_n}(t) & = E\left[e^{tT_n}\right]                                                             \\&= E\left[e^{\frac{t}{\sqrt{n}}Y_1+\cdots+Y_n}\right] \\
                      & =E\left[e^{\frac{t}{\sqrt{n}}Y}\right]^n                                             \\
                      & = E \left[ 1 + \frac{t}{\sqrt{n}}Y + \frac{t^2}{2n}Y^2 + \cdots \right]^n            \\
                      & =E\left[1+0+\frac{t^2}{2n}Y^2\right]^n                                               \\
                      & = \left(1 + \frac{t^2}{2n} E[Y^2]\right)^n \\
                      &=(1+\frac{t^2}{2n})\tag*{since $E[Y^2] = Var[Y] + (E[Y])^2$}^n  \\
                      &=e^{\frac{t^2}{2}} \quad \text{as } n \to \infty \\
                      & = \phi_Z(t)
    \end{align*}
\end{proof}

\section{Markov Chain}