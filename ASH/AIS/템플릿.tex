% 이 문서는 FSU의 MATH 2300-04 수강생들에게 LaTeX과 Overleaf를 소개하기 위해 작성되었습니다. 다른 학생들도 자유롭게 사용해도 됩니다.

% 앞에 '%'가 있는 모든 내용은 주석이며 컴파일러가 무시합니다.
%
% "\begin{document}" 줄 앞의 내용을 프리앰블(preamble)이라고 합니다.
% 여기서 필요한 특정 패키지를 불러옵니다.
% 더 숙달되거나 프로그램에서 패키지가 없다는 메시지가 나오기 전까지는 무시해도 안전합니다.
%
%----------------------------------

\documentclass[12pt]{article}
\usepackage[margin=1in]{geometry}% 원한다면 여기서 여백을 변경하세요.
\setlength{\parindent}{0pt} % 이건 새 문단의 들여쓰기 길이를 설정합니다. 원하면 변경하세요.
\setlength{\parskip}{5pt} % 이건 문단 간의 거리를 설정하며, LaTeX 코드에 빈 줄이 있을 때마다 사용됩니다.
\pagenumbering{gobble}% 이건 페이지 번호를 매기지 않는다는 뜻입니다. 페이지 번호를 원하면 주석 처리하세요.


%이 패키지들은 대부분의 일반적인 "수학적 기능"을 가능하게 합니다.
\usepackage{amsmath,amsthm,amssymb}

%이 패키지는 그래프나 다른 이미지를 추가할 수 있게 해줍니다.
\usepackage{graphicx}

%이것들은 제가 주로 사용하며 이 문서에 필요한 패키지들입니다. 원하는 거의 모든 것을 할 수 있는 수많은 다른 패키지들이 있습니다.
\usepackage{color}
\usepackage{enumerate}
\usepackage{multicol}
\usepackage{kotex}



%------------------------------------------
% 편집하려는 내용은 여기서부터 시작됩니다.
%------------------------------------------
\begin{document}

\title{Sample \LaTeX \, Document} % 이걸 프로젝트의 제목으로 설정해야 합니다.
\author{Sarah Wright} % 이걸 본인의 이름으로 바꾸세요.
% \date{} 프로그램이 오늘 날짜를 사용하도록 주석 처리했습니다. {} 안에 특정 날짜를 지정하거나(또는 필요한 경우 다른 정보를 넣거나), 날짜(또는 기타 정보)가 제목에 나타나지 않게 하려면 {}를 비워둘 수 있습니다.

\maketitle % 격식 있는 제목을 원하지 않으면 이 부분과 위 내용을 지울 수 있습니다. 동일한 정보를 포함할 다른 방법을 찾아보세요.

% 어떤 프로젝트는 단순히 문제 목록일 수 있으며, 그런 경우 번호를 매기고 싶을 수 있습니다:

\begin{enumerate} % 이 환경은 목록의 "\items"에 1부터 번호를 매깁니다. 번호 매기기 방식을 바꾸려면 \begin{enumerate}[(a)]처럼 할 수 있으며, [] 안의 내용이 "번호"의 형식이 됩니다. 특정 항목의 번호만 바꾸려면 \item[1.2.7]을 사용하세요.
\item Use the formal definition of the limit of a function at a point to prove that the following holds:
$$\lim_{x \rightarrow 44} x^2 + x - 5$$
% 모든 수식은 $ 안에 들어갑니다. 양쪽에 $ 하나씩 있으면 수식이 텍스트 줄 안에 포함됩니다. 양쪽에 $ 두 개씩 있으면 수식은 페이지 중앙의 별도 줄에 배치됩니다.


\begin{proof} % \begin{}이 있는 모든 것은 대응되는 \end{}가 필요합니다. 괄호 {} 안에 들어가는 것이 작업 중인 환경입니다. 이것은 증명(proof)이며, 이 환경은 컴파일러에게 시작 부분에 이탤릭체로 "Proof"를 쓰고 끝부분에 상자를 표시하도록 지시합니다. 이 미적분학 수업에서 증명을 많이 하지는 않겠지만, 알아두면 좋습니다.

Fix an arbitrary $\epsilon > 0$.% $ 한 쌍 안에 있는 수식을 주목하세요. 대부분의 그리스 문자는 앞에 \를 붙여서 씁니다; 대문자 그리스 문자를 원하면 첫 글자를 대문자로 쓰세요.

We wish to determine a $\delta >0$ such that when $0 < |x - 4| < \delta$, it must be true that $|f(x) - 15| < \epsilon$.

\begin{multicols}{2} % 이 환경은 다단으로 작성할 수 있게 해주며, 때때로 유용합니다. 가끔 까다로울 수 있으니 주의하세요. {} 안의 숫자는 원하는 단의 개수입니다.
Choose $\displaystyle{\delta = \min\left\{1, \frac{\epsilon}{10}\right\}}$. % 이 한 줄에서 많은 일이 일어나고 있습니다:
%
% \displaystyle{}은 괄호 안의 수식을 이중 $$와 동일하게 처리하지만, 텍스트 줄 안에 유지합니다. 분수, 극한, 적분 등 "키가 큰" 기호에 좋습니다.
%
% \left와 \right 뒤에 괄호 기호 (, [, < 등이 오면 내부 수식에 맞는 적절한 크기로 기호를 만듭니다. 모든 \left는 \right와 짝을 이뤄야 하지만, 기호가 일치할 필요는 없어서 $\left(3, 4\right]$도 컴파일됩니다.
%
% 코드에서 {}가 자주 사용되므로, 컴파일된 문서에 {}를 표시하려면 앞에 \를 붙여야 합니다.
%
% 분수는 \frac{}{}을 사용하여 만듭니다. 분자 식은 첫 번째 중괄호 {} 세트에, 분모는 두 번째에 넣습니다.

Now, suppose that $0 < |x - 4| < \delta$.  Then,
\begin{align*} % 또 다른 환경입니다! align*은 작업물 속의 &를 수직 열로 정렬해 줍니다. 긴 계산에서 등호를 맞추는 데 좋습니다. *를 빼면 각 줄에 번호가 매겨집니다. 나중에 다시 참조할 때 유용할 수 있습니다.
|f(x) - 15| & = |(x^2 + x - 5) - 15| \text{, by the definition of $f$,}\\ % \\는 align* 환경에게 다음 줄로 이동하라고 지시합니다. align은 강제로 수식 환경을 적용하므로, align 내부에서 "일반" 텍스트를 원할 때는 명시해줘야 합니다.
& = |x^2 + x - 15|\\
& = |(x - 4)(x + 5)|\\
& = |x - 4||x + 5| \text{, by properties of absolute value,}\\
& < \delta \cdot |x + 5| \text{, by the assumption $|x - 4| < \delta$,}\\
& \leq \frac{\epsilon}{10}|x + 5| \text{, since $\delta \leq \frac{\epsilon}{10}$,}\\
& = \frac{\epsilon}{10} |(x - 4) + 9|\\
& \leq \frac{\epsilon}{10}\left(|x - 4| + |9|\right) \text{, by properties of absolute value,}\\
& < \frac{\epsilon}{10}\left(\delta + 9\right) \text{, since $|x - 4| < \delta$,}\\
& \leq \frac{\epsilon}{10}(1 + 9) \text{, since $\delta \leq 1$,}\\
& = \left(\frac{\epsilon}{10}\right) (10) = \epsilon
\end{align*}

\columnbreak % 이것은 단(column)을 끝냅니다. 때로는 원하는 곳에서 자동으로 발생하기도 하고, 그렇지 않기도 합니다.

{\color{blue} %이것은 {} 안의 모든 색상을 파란색으로 만듭니다(오른쪽 컴파일된 pdf에서). \LaTeX은 색상 목록을 알고 있으니, 여러 가지를 시도해보거나 목록을 구글링해보세요.
\center{\bf Scratch Work} % \center는 {} 안의 모든 내용을 가운데 정렬하고 \bf는 텍스트를 굵게 만듭니다.
\begin{align*} % 여기 \hspace{}는 두 번째 열의 내용을 조금 이동시키기 위한 꼼수입니다. 지우고 무슨 일이 일어나는지 보세요. \vspace{}나 \hspace{}를 사용하여 작업물에 수직 또는 수평 공백을 추가할 수 있습니다. LaTeX의 자동 서식 지정 때문에 항상 예상대로 작동하지는 않습니다.
\hspace{2in}|f(x) - 15| & < \epsilon \\ 
|(x^2 + x - 5) - (15)| & < \epsilon \\
|x^2 + x - 20| & < \epsilon \\
|(x + 5)(x - 4)| & < \epsilon \\
|(x - 4)|\cdot|(x + 5)| & < \epsilon \\
|x - 4| & < \frac{\epsilon}{|x + 5|} \\ 
\end{align*}
\begin{align*}
\hspace{1.75in}\delta = 1 \Longrightarrow |x - 4| & < 1\\
-1 < x - 4 & < 1\\
8 < x + 5 & < 10\\
| x + 5| & < 10
\end{align*}
}
\end{multicols}
All together, this shows that for any $\epsilon >0$, if we choose $\displaystyle{\delta = \min\left\{1, \frac{\epsilon}{10}\right\}}$, then $0 \leq |x - 4| < \delta$ implies that $|f(x) - 15| < \epsilon$.  Thus, $\displaystyle{\lim_{x \rightarrow 4} x^2 + x - 5 = 15}$.

\end{proof}

\newpage % 이것은 새 페이지를 시작합니다.
% 저는 여기서 이게 필요 없었습니다. LaTeX은 때때로 스스로 알아서 할 만큼 똑똑합니다. 다음 문제의 한 줄만 입력했을 때는 첫 페이지 하단에 나타나서 마음에 안 들었습니다. 하지만 내용을 더 추가하자 LaTeX이 알아서 옮겼습니다.
% 재미있는 사실! 작업 입력을 마친 후 이 위치를 찾아 주석을 추가해야 했을 때, 오른쪽 pdf에서 원하는 위치를 클릭하니 Overleaf가 LaTeX 코드의 이 위치로 데려다주었습니다. 꽤 멋지죠.

\item[1.5.15] Evaluate the given limits of the piecewise defined function $f$.
$$f(x) = \left\{\begin{array}{lcl}
x^2 - 1 & \text{ if }& x < -1 \\
x^3 + 1 & \text{ if } & -1 \leq x \leq 1 \\
x^2 + 1 & \text{ if } & x > 1
\end{array}\right.$$
% array는 행렬을 위해 설계된 또 다른 환경이지만, 조각적으로 정의된 함수에도 잘 작동합니다. 여기 {lcl}은 첫 번째와 마지막은 왼쪽 정렬, 가운데는 중앙 정렬인 3개의 열을 원한다고 LaTeX에 알립니다. &를 사용하여 새 열을 나타내고 \\는 새 행/줄로 이동합니다. align처럼 수식 환경이므로 $는 필요 없지만 텍스트는 표시해줘야 합니다.


\begin{enumerate} % 여러 부분이 있는 문제나 풀이를 위해 enumerate 환경을 중첩할 수 있습니다.
\item $\displaystyle{\lim_{x \rightarrow -1^-} f(x)}$

Since we are evaluating the limit as $x$ approaches -1 from the left, we need to consider the form of the function for values of $x$ that are less than -1, $x^2 - 1$.
\begin{align*}
\lim_{x \rightarrow -1^-} f(x) & = \lim_{x \rightarrow -1^-} x^2 - 1\\
& = (-1)^2 - 1 \text{, by Theorem 2,}\\
& = 0.
\end{align*}

\bigskip %이것은 수직 공백을 추가하는 또 다른 방법입니다. big, med, small 종류가 있습니다.

\item $\displaystyle{\lim_{x \rightarrow -1^+} f(x)}$

Since we are evaluating the limit as $x$ approaches -1 from the right, we need to consider the form of the function for values of $x$ that are greater than -1, $x^3 + 1$.
\begin{align*}
\lim_{x \rightarrow -1^+} f(x) & = \lim_{x \rightarrow -1^+} x^3 + 1\\
& = (-1)^3 + 1 \text{, by Theorem 2,}\\
& = 0.
\end{align*}

\bigskip

\item $\displaystyle{\lim_{x \rightarrow -1} f(x)}$

Since $\displaystyle{\lim_{x \rightarrow -1^-} f(x) = \lim_{x \rightarrow -1^+} f(x) = 0}$, $\displaystyle{\lim_{x \rightarrow -1} f(x) = 0}$ by Theorem 7.

\bigskip

\item $f(-1)$

When $x = -1$, $f(x) = x^3 + 1$.  So, $f(-1) = (-1)^3 + 1 = 0$.

\bigskip

\item $\displaystyle{\lim_{x \rightarrow 1^-} f(x)}$

Since we are evaluating the limit as $x$ approaches 1 from the left, we need to consider the form of the function for values of $x$ that are less than (but near) 1, $x^3 + 1$.
\begin{align*}
\lim_{x \rightarrow 1^-} f(x) & = \lim_{x \rightarrow 1^-} x^3 + 1\\
& = (1)^3 + 1 \text{, by Theorem 2,}\\
& = 2.
\end{align*}

\bigskip

\item $\displaystyle{\lim_{x \rightarrow 1^+} f(x)}$

Since we are evaluating the limit as $x$ approaches 1 from the right, we need to consider the form of the function for values of $x$ that are greater than (but near) 1, $x^2 + 1$.
\begin{align*}
\lim_{x \rightarrow 1^-} f(x) & = \lim_{x \rightarrow 1^+} x^2 + 1\\
& = (1)^2 + 1 \text{, by Theorem 2,}\\
& = 2.
\end{align*}

\bigskip

\item $\displaystyle{\lim_{x \rightarrow 1} f(x)}$

Since $\displaystyle{\lim_{x \rightarrow 1^-} f(x) = \lim_{x \rightarrow 1^+} f(x) = 2}$, $\displaystyle{\lim_{x \rightarrow 1} f(x) = 2}$ by Theorem 7.

\bigskip

\item $f(1)$

When $x = 1$, $f(x) = x^3 + 1$.  So, $f(1) = (1)^3 + 1 = 2$.

\end{enumerate}

\vfill % 이것은 수직 공백을 추가하는 또 다른 방법이며, 유사한 수평 옵션도 있습니다. 공간을 채워주며, 여러 개를 사용하여 페이지에 내용을 고르게 배치할 수 있습니다.

To help us visualize all of these limits, a graph of $y = f(x)$ is provided below.

% \begin{center}\includegraphics[width = .85\textwidth]{SampleGraph}\end{center}
% 이것이 LaTeX 문서에 그림을 포함하는 방법입니다. Overleaf에서는 프로젝트에 이 파일을 추가해야 합니다. 상단 메뉴의 사각형이 있는 "Project"를 클릭하세요. "Add Files..."를 선택하고 컴퓨터에서 업로드하면 됩니다. (개인 컴퓨터에 LaTeX 편집기를 설치한 경우, 포함하려는 파일이 .tex 문서와 같은 폴더에 있는지 확인하거나... 다른 폴더를 호출하는 방법을 찾아보세요.)
% 파일이 프로젝트의 일부가 되면 {} 안에 파일 이름을 입력하세요. 보통 .pdf, .jpg 등의 파일 형식은 필요 없지만 포함해도 문제는 없습니다.
% []는 비워두거나 다양한 지침을 입력하는 데 사용할 수 있으며, 이미지 크기 조정이 가장 일반적입니다. 인치, 센티미터, 포인트를 사용할 수 있으며 LaTeX은 많은 측정 시스템을 알고 있습니다. 위에서 한 것처럼 상대적인 길이를 지정할 수도 있습니다. \textwidth는 이 문서의 텍스트 너비이며, 앞에 .85를 붙이면 텍스트 너비의 85%로 줄어듭니다.

\end{enumerate}


\end{document}