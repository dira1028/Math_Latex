\documentclass{article}
\usepackage{amsmath}    % 수식 작성을 위한 핵심 패키지 (align, gather 등)
\usepackage{amssymb}    % 추가 수학 기호 사용을 위한 패키지
\usepackage{kotex}      % 한글 사용을 위한 패키지
\usepackage{amsthm}     % 정리/예제 환경 정의를 위한 패키지
\usepackage{geometry}   % 페이지 여백 설정을 위한 패키지
\usepackage{tcolorbox}  % 박스 코멘트 작성을 위한 패키지

% amsthm: 번호 없는 예제 환경 정의
\newtheorem*{ex}{예제} 

% tcolorbox: 박스 코멘트 환경 정의
\newtcolorbox{mycomment}{
    colback=yellow!10!white,    % 박스 배경색 (밝은 노랑)
    colframe=red!50!black,      % 박스 테두리색 (붉은 검정)
    boxsep=5pt,                 
    arc=4pt,                    
    leftrule=0.5mm, rightrule=0.5mm, toprule=0.5mm, bottomrule=0.5mm 
}


\title{미분방정식}
\author{}
\date{}

\begin{document}

\maketitle

\section*{미분방정식의 분류}

\subsection*{선형 방정식 (Linear Equations)}
\begin{itemize}
    \item $y' + y = 0$
    \item $x^2y' + y = 0$
\end{itemize}

\subsection*{비선형 방정식 (Non-linear Equations)}
\begin{itemize}
    \item $yy' + y = 0$
    \item $y' = y^2$
\end{itemize}

\subsection*{정규형 (Normal Form)}
\begin{itemize}
    \item $ f(x, y, y') = 0 \Rightarrow y' = f(x, y) $
\end{itemize}

\begin{mycomment}
    미분방정식의 해를 구할 수 없는 경우에는 Slope Field를 통해 Solution Curve를 추정한다.
\end{mycomment}

\clearpage

\section{Separable Equations}

Separable Equation의 형태는 다음과 같다
\[ \frac{dy}{dx} = H(x, y) = g(x)h(y) \]
\[ \frac{dy}{dx} = \frac{g(x)}{f(y)} \quad \text{or} \quad f(y)\frac{dy}{dx} - g(x) = 0 \]

\subsection{기본 풀이법}

\begin{ex} 
다음 미분 방정식을 풀어보시오.
\[ \frac{dy}{dx} = -6xy \]
\end{ex}

\textbf{Pf)}
\begin{enumerate}
    \item x, y에 대해 항 분리
    \begin{equation} \label{eq:separate}
        \frac{1}{y}dy = -6x dx
    \end{equation}
    
    \item 양변을 적분
    \begin{align}
        \int \frac{1}{y}dy &= \int -6x dx \notag \\ 
        \ln|y| &= -3x^2 + C \tag{*}
    \end{align}
    \item $y$에 대해 정리
    \begin{equation}
        y = Ae^{-3x^2} 
    \end{equation}
\end{enumerate}


\subsection*{(1)번과 (2)번 과정으로 넘어갈 수 있는 이유} 

\textbf{how (1)?}
\begin{align*}
    \frac{dy}{dx} = \frac{g(x)}{f(y)} &\to \frac{1}{f(y)}dy = g(x)dx \\
    &\int \frac{1}{f(y)}dy = \int g(x)dx \tag{**}\\ 
\end{align*}

\begin{align*}
    f(y)\frac{dy}{dx} - g(x) = 0 \\
    \intertext{}\\
    \intertext{Let $F'(y) = f(y)$ and $G'(x) = g(x)$} % F'(x)를 F'(y)로 수정했습니다.
    \frac{d}{dx}[F(y(x))] &=F'(y) \cdot \frac{dy}{dx}=f(y)\frac{dy}{dx} \quad \text{(by Chain Rule)} \\
    &\to \frac{d}{dx}[F(y(x))] - \frac{d}{dx}[G(x)] = 0 \\
    &\to \frac{d}{dx}[F(y(x)) - G(x)] = 0 \\
    &\to F(y(x)) - G(x) = C \quad \\
    &= \int f(y) dy = \int g(x) dx
\end{align*}

이는 $\text{(**)}$의 결과와 동일하다.


\noindent\textbf{how (2)?}
\begin{align*}
    \int \frac{1}{y}dy &= \begin{cases} 
        \ln(y) + C_1& \text{if } y > 0 \\ 
        \ln(-y) + C_2& \text{if } y < 0 
    \end{cases}
\end{align*}

$\text{(*) $\to$ (2)}$로 넘어갈 수 있는 이유는 실제 미분방정식의 해를 통해 Solution Curve를 구할 때\\
IC에 의해 $y > 0$인 경우와 $y < 0$인 경우 둘 중 하나로 정해진다. \\
따라서 하나의 일반해로 표기한다.


\end{document}